Um die genannten Punkte des letzten Kapitels praktisch in die Tat umzusetzen, müssen bestimmte Vorbereitungen getroffen werden. Zunächst muss evaluiert werden, ob TOGAF als neue Architekturmethode für die bestehenden Teams zu der bestehenden Zeit der Migration überhaupt in Frage kommt. Faktoren, die für diese Evaluation eine Rolle spielen, sind u.a. der jetzige Stand der Migration, die Bereitschaft der Mitarbeiter einer grundlegenden Arbeitsanpassung, sowie der geschätzte Aufwand, bestehende Prozesse auf TOGAF umzustrukturieren. Stellt sich heraus, dass die Umstrukturierung realisierbar und sinnvoll ist, müssten Schulungen der Mitarbeiter und eben genannte Prozessveränderungen getätigt werden, bevor die Arbeit am eigentlichen Projekt wieder aufgenommen werden kann. Die benötigten Dokumentationen müssten ebenfalls nachträglich angelegt werden, um nahtlos innerhalb der Prozesses an die folgenden Schritte anknüpfen zu können. Sofern diese Fragestellungen geklärt sind, bedarf die praktische Umsetzung der Architecture Governance in jedem Fall die Absprache mit der Rechtsabteilung der Bank, vor allem mit Betrachtung auf die definierte Regelung der Ausnahmegenehmigungen. 