Auch für die Migration auf die Soll-Architektur (oder auch \textit{Target Architecture}) sieht das ADM von TOGAF eine eigene Phase vor \cite{TOGAFDocs}. So stellt Phase F das Migration Planning dar und knüpfen somit direkt an die Evaluationsphase E "Opportunities and Solutions" an. Phase F sieht die Erstellung einer sog. Architecture Roadmap vor, welche die tatsächliche Implementierung unterstützt. Dieser Prozess versichert, dass die geplante Migration entsprechend der unternehmensspezifischen Vorgaben und Ziele durchgeführt wird, und dass die Faktoren Wert und Geld, welche im Rahmen dieser Aufgabestellung von großer Bedeutung sind, von allen Beteiligten nachvollzogen werden können \cite{VisualParadigmTOGAF}. Die Aufgabenstellung nimmt ebenfalls an, dass Entscheidungen rund um die Bestandteile und deren Realisierbarkeit bereits getroffen wurden, weswegen die Vorbereitungsschritte der Phase F hier vernachlässigt werden können.

Im Folgenden wird eine Reihe von Tätigkeiten für Application Owner nach TOGAF ADM vorgeschlagen, welche als Checkliste (s. Anhang \ref{app:checkliste}) für den praktischen Migrationsprozess genutzt werden können.

Da es sich oftmals um einen Migrationsplan handeln kann, welcher aus mehreren Teilschritten besteht, müssen diese Teilprojekte für die praktische Umsetzung priorisiert werden. Dazu erfolgen Bemessungen und Bewertungen der einzelnen Projekte nach unterschiedlichen Gesichtspunkten, u.a. kritischen Erfolgsfaktoren oder definierten Kriterien für Return on Investment.

Zwar sind die Zielressourcen definiert, jedoch ist die Verfügbarkeit und deren Implementierungsanforderungen eventuell nicht geklärt. Dies muss entsprechend geschehen und in die Priorisierung mit einfließen.

Ein weiteres Kriterium ist die Betrachtung der einzelnen Risikofaktoren der unterschiedlichen Teilprojekte, wobei sich dafür eine Szenariomethode eignen kann.

Mit diesen Eingaben und den entstehenden Prioritäten kann die besagte Architecture Roadmap angelegt und dokumentiert werden, welche die praktische Migration zeitlich und logistisch definiert.

Da TOGAF auch für nicht-technische Architekturen angewendet werden kann, sind die o.g. Punkte sehr abstrakt formuliert. Um die Priorisierung dennoch vollführen zu können, unterstützen folgende Fragestellungen über Komplexität und Dringlichkeit: Wie wurden die Anwendungen entwickelt, sind die Dokumentationen verfügbar und wie viele Unternehmensbereiche werden von der Migration beeinflusst? Darüber hinaus kann ebenfalls betrachtet werden, wie sehr der Unternehmenserfolg von der Anwendungsmigration abhängig ist und wie diese verwaltet wird.

Zusammengefasst können diese Schritte dazu verwendet werden, um einen prioritätsgetreuen Handlungsablauf zu planen, welcher die Migration des Systems auf sämtlichen Ebenen plant.