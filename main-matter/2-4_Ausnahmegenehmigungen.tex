Wenn eine Anwendung die Anforderungen, die durch die Architektur gegeben sind noch nicht erfüllt, aber trotzdem zeitnah bereitgestellt werden muss, gibt es einen Prozess um eine eingeschränkte Ausnahmegenehmigung zu erhalten. Eine Anwendung kann solch eine Ausnahmegenehmigung erhalten, wenn durch technische Hürden aktuell keine Compliance erreicht werden kann, der Compliance Prozess eine Geschäftsentscheidende Anwendung aufhält, wodurch substantielle Verluste zu erwarten sind, oder wenn die Enterprise Architektur nicht auf die Anwendung anwendbar ist. 

Mindestens einer der genannten Fälle muss nach Prüfung des Einzelfalles zutreffend sein, damit eine Ausnahmegenehmigung erteilt wird. In allen Fällen muss die Anwendung den vollständigen Compliance Prozess jedoch zu einem späteren Zeitpunkt, der bei der Erteilung einer Ausnahmegenehmigung festesetzt wird, durchlaufen. 

Sollte sich bei der Prüfung des Antrags auf Ausnahmegenehmigung herausstellen, dass keiner der genannten Gründe erfüllt ist, muss ein Beratungstermin mit dem verantwortlichen Architekten oder Implementierungsteam stattfinden, der zur Aufklärung dient. Dies soll die Anzahl an nicht notwendigen Anträgen minimieren. 

Sollte sich bei der Prüfung herausstellen, dass die Enterprise Architektur nicht auf die Anwendung anwendbar ist, muss ein entsprechender Prozess zur Erweiterung oder Anpassung der Architektur angestossen werden und der Termin für Compliance Prüfung der entsprechenden Anwendung nach dem Abschluss dieses Prozesses stattfinden.

Für Ausnahmen von der Architektur wird ein 5-Phasen Modell vorgeschlagen \cite{exception-governance}:

\begin{enumerate}
\item Ausnahmezustand: Eine Ausnahme von der Architektur besteht
\item Gelöst: Die Ausnahme wurde beseitigt
\item Aufgelöst: Die Lösung der Ausnahme wurde vom Application Owner abgesegnet
\item Archiviert: Die Ausnahme bleibt für Dokumentationszwecke archiviert
\item Gelöscht: Die Ausnahme ist gelöscht und nicht länger in der Dokumentation aufzufinden
\end{enumerate}

Regel: Anwendungen dürfen, wenn überhaupt, nur temporär ohne erfolgreiche Compliance Prüfung bereitgestellt werden.
