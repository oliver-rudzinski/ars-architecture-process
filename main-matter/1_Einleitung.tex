Die 17A-Bank ist zur Zeit in vier Ländern aktiv (Deutschland, Slowenien, Tschechien und Österreich) und managt jedes davon unabhängig. Jedes Land hat dabei eine unabhängige IT-Infrastruktur, was sich als zunehmend problematisch gestaltet. Anwendungen und Daten liegen redundant und verstreut vor. In den vier Rechenzentren befindet sich ein Sammelsurium an Hard- und Software. Wartungsarbeiten können sich oft als komplex und teuer erweisen. Das alles führt nicht nur zu hohen Kosten, sondern macht es auch schwer bis unmöglich Daten zusammenzuführen um Beispielsweise ein Gesamtbild vom Kunden und seinem verhalten zu gewinnen. 

Zur Lösung dieser Probleme sollen die IT-Standorte und Rechenzentren zusammengeführt werden. Die vier Rechenzentren sollen dabei zu einem zentralen in Frankfurt konsolidiert werden. Die diversen Anwendungen für Girokonten, Sparkonten, Kredite und Kreditkarten sollen zukünftig auch konsolidiert werden. Aufgrund des hohen Aufwands einer Konsolidierung auf Anwendungsebene, hat sich die 17A-Bank für eine Konsolidierung auf Infrastrukturebene entschieden. Die Rechenzentren sollen also in ein modernes Rechenzentrum in Frankfurt zusammengelegt werden. Es sind bereits Standardplattformen für Storage und Server aufgebaut, sowie Standards für Betriebssysteme und Softwarestacks definiert. Das Rechenzentrum wird bei einem Hosting-Anbieter angemietet.

Um diesen Prozess sinnvoll und für die Mitarbeiter verständlich zu gestalten, sowie zukünftige Anwendungen besser in die bestehenden IT-Lösungen zu integrieren bedarf es eines Architekturprozesses.

Dieser soll eine Reihe von Anforderungen erfüllen:

Der Architekturprozess muss zunächst eine gute Zusammenarbeit zwischen Entwicklungsteams und Infrastrukturteams ermöglichen. Infrastrukturarchitekten sollen frühzeitig in die Migrierungsprozesse involviert werden können. Gegebenenfalls soll er auch ein Deployment in eine Public Cloud ermöglichen. Er muss es ermöglichen die Anforderungen an Anwendungen flexibel zu definieren und umzusetzen.

Zusätzlich soll eine Checkliste für Application Owner, die in das neue Datacenter migrieren werden, ein Architecture-Governance Prozess und ein Genehmigungsprozess für Ausnahmen von der Architektur dargelegt werden.
